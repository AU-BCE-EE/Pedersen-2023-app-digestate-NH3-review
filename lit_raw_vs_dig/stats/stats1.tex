% Options for packages loaded elsewhere
\PassOptionsToPackage{unicode}{hyperref}
\PassOptionsToPackage{hyphens}{url}
%
\documentclass[
]{article}
\usepackage{amsmath,amssymb}
\usepackage{iftex}
\ifPDFTeX
  \usepackage[T1]{fontenc}
  \usepackage[utf8]{inputenc}
  \usepackage{textcomp} % provide euro and other symbols
\else % if luatex or xetex
  \usepackage{unicode-math} % this also loads fontspec
  \defaultfontfeatures{Scale=MatchLowercase}
  \defaultfontfeatures[\rmfamily]{Ligatures=TeX,Scale=1}
\fi
\usepackage{lmodern}
\ifPDFTeX\else
  % xetex/luatex font selection
\fi
% Use upquote if available, for straight quotes in verbatim environments
\IfFileExists{upquote.sty}{\usepackage{upquote}}{}
\IfFileExists{microtype.sty}{% use microtype if available
  \usepackage[]{microtype}
  \UseMicrotypeSet[protrusion]{basicmath} % disable protrusion for tt fonts
}{}
\makeatletter
\@ifundefined{KOMAClassName}{% if non-KOMA class
  \IfFileExists{parskip.sty}{%
    \usepackage{parskip}
  }{% else
    \setlength{\parindent}{0pt}
    \setlength{\parskip}{6pt plus 2pt minus 1pt}}
}{% if KOMA class
  \KOMAoptions{parskip=half}}
\makeatother
\usepackage{xcolor}
\usepackage[margin=1in]{geometry}
\usepackage{color}
\usepackage{fancyvrb}
\newcommand{\VerbBar}{|}
\newcommand{\VERB}{\Verb[commandchars=\\\{\}]}
\DefineVerbatimEnvironment{Highlighting}{Verbatim}{commandchars=\\\{\}}
% Add ',fontsize=\small' for more characters per line
\usepackage{framed}
\definecolor{shadecolor}{RGB}{248,248,248}
\newenvironment{Shaded}{\begin{snugshade}}{\end{snugshade}}
\newcommand{\AlertTok}[1]{\textcolor[rgb]{0.94,0.16,0.16}{#1}}
\newcommand{\AnnotationTok}[1]{\textcolor[rgb]{0.56,0.35,0.01}{\textbf{\textit{#1}}}}
\newcommand{\AttributeTok}[1]{\textcolor[rgb]{0.13,0.29,0.53}{#1}}
\newcommand{\BaseNTok}[1]{\textcolor[rgb]{0.00,0.00,0.81}{#1}}
\newcommand{\BuiltInTok}[1]{#1}
\newcommand{\CharTok}[1]{\textcolor[rgb]{0.31,0.60,0.02}{#1}}
\newcommand{\CommentTok}[1]{\textcolor[rgb]{0.56,0.35,0.01}{\textit{#1}}}
\newcommand{\CommentVarTok}[1]{\textcolor[rgb]{0.56,0.35,0.01}{\textbf{\textit{#1}}}}
\newcommand{\ConstantTok}[1]{\textcolor[rgb]{0.56,0.35,0.01}{#1}}
\newcommand{\ControlFlowTok}[1]{\textcolor[rgb]{0.13,0.29,0.53}{\textbf{#1}}}
\newcommand{\DataTypeTok}[1]{\textcolor[rgb]{0.13,0.29,0.53}{#1}}
\newcommand{\DecValTok}[1]{\textcolor[rgb]{0.00,0.00,0.81}{#1}}
\newcommand{\DocumentationTok}[1]{\textcolor[rgb]{0.56,0.35,0.01}{\textbf{\textit{#1}}}}
\newcommand{\ErrorTok}[1]{\textcolor[rgb]{0.64,0.00,0.00}{\textbf{#1}}}
\newcommand{\ExtensionTok}[1]{#1}
\newcommand{\FloatTok}[1]{\textcolor[rgb]{0.00,0.00,0.81}{#1}}
\newcommand{\FunctionTok}[1]{\textcolor[rgb]{0.13,0.29,0.53}{\textbf{#1}}}
\newcommand{\ImportTok}[1]{#1}
\newcommand{\InformationTok}[1]{\textcolor[rgb]{0.56,0.35,0.01}{\textbf{\textit{#1}}}}
\newcommand{\KeywordTok}[1]{\textcolor[rgb]{0.13,0.29,0.53}{\textbf{#1}}}
\newcommand{\NormalTok}[1]{#1}
\newcommand{\OperatorTok}[1]{\textcolor[rgb]{0.81,0.36,0.00}{\textbf{#1}}}
\newcommand{\OtherTok}[1]{\textcolor[rgb]{0.56,0.35,0.01}{#1}}
\newcommand{\PreprocessorTok}[1]{\textcolor[rgb]{0.56,0.35,0.01}{\textit{#1}}}
\newcommand{\RegionMarkerTok}[1]{#1}
\newcommand{\SpecialCharTok}[1]{\textcolor[rgb]{0.81,0.36,0.00}{\textbf{#1}}}
\newcommand{\SpecialStringTok}[1]{\textcolor[rgb]{0.31,0.60,0.02}{#1}}
\newcommand{\StringTok}[1]{\textcolor[rgb]{0.31,0.60,0.02}{#1}}
\newcommand{\VariableTok}[1]{\textcolor[rgb]{0.00,0.00,0.00}{#1}}
\newcommand{\VerbatimStringTok}[1]{\textcolor[rgb]{0.31,0.60,0.02}{#1}}
\newcommand{\WarningTok}[1]{\textcolor[rgb]{0.56,0.35,0.01}{\textbf{\textit{#1}}}}
\usepackage{graphicx}
\makeatletter
\def\maxwidth{\ifdim\Gin@nat@width>\linewidth\linewidth\else\Gin@nat@width\fi}
\def\maxheight{\ifdim\Gin@nat@height>\textheight\textheight\else\Gin@nat@height\fi}
\makeatother
% Scale images if necessary, so that they will not overflow the page
% margins by default, and it is still possible to overwrite the defaults
% using explicit options in \includegraphics[width, height, ...]{}
\setkeys{Gin}{width=\maxwidth,height=\maxheight,keepaspectratio}
% Set default figure placement to htbp
\makeatletter
\def\fps@figure{htbp}
\makeatother
\setlength{\emergencystretch}{3em} % prevent overfull lines
\providecommand{\tightlist}{%
  \setlength{\itemsep}{0pt}\setlength{\parskip}{0pt}}
\setcounter{secnumdepth}{-\maxdimen} % remove section numbering
\ifLuaTeX
  \usepackage{selnolig}  % disable illegal ligatures
\fi
\IfFileExists{bookmark.sty}{\usepackage{bookmark}}{\usepackage{hyperref}}
\IfFileExists{xurl.sty}{\usepackage{xurl}}{} % add URL line breaks if available
\urlstyle{same}
\hypersetup{
  pdftitle={Stats 1: Effect of digestion on composition and more},
  pdfauthor={Sasha D. Hafner},
  hidelinks,
  pdfcreator={LaTeX via pandoc}}

\title{Stats 1: Effect of digestion on composition and more}
\author{Sasha D. Hafner}
\date{30 august, 2023}

\begin{document}
\maketitle

\hypertarget{vars}{%
\section{Vars}\label{vars}}

Difference in pH due to digestion.

\begin{Shaded}
\begin{Highlighting}[]
\NormalTok{dw}\SpecialCharTok{$}\NormalTok{dpH }\OtherTok{\textless{}{-}}\NormalTok{ dw}\SpecialCharTok{$}\NormalTok{pH.dig }\SpecialCharTok{{-}}\NormalTok{ dw}\SpecialCharTok{$}\NormalTok{pH.ref}
\NormalTok{dw}\SpecialCharTok{$}\NormalTok{dDM }\OtherTok{\textless{}{-}}\NormalTok{ dw}\SpecialCharTok{$}\NormalTok{DM.dig }\SpecialCharTok{{-}}\NormalTok{ dw}\SpecialCharTok{$}\NormalTok{DM.ref}
\end{Highlighting}
\end{Shaded}

\begin{Shaded}
\begin{Highlighting}[]
\FunctionTok{table}\NormalTok{(dw}\SpecialCharTok{$}\NormalTok{source)}
\end{Highlighting}
\end{Shaded}

\begin{verbatim}
## 
##      Amon et al. (2006)  Anderson et al. (2023) Chantigny et al. (2007) Chantigny et al. (2009)   Clemens et al. (2006) 
##                       1                       1                       3                       3                       3 
##    Hansen et al. (2004)    Hjorth et al. (2009)     Lemes et al. (2023)    Möller et al. (2009) Neerackal et al. (2015) 
##                       2                       4                       2                       2                       4 
##     Nyord et al. (2012)  Pedersen et al. (2021)     Rubæk et al. (1996)    Sommer et al. (2006)    Wagner et al. (2021) 
##                       1                       2                       4                       2                       2 
##      Wulf et al. (2002) 
##                       2
\end{verbatim}

\hypertarget{plots}{%
\section{Plots}\label{plots}}

\begin{Shaded}
\begin{Highlighting}[]
\FunctionTok{ggplot}\NormalTok{(dat, }\FunctionTok{aes}\NormalTok{(source, pH, }\AttributeTok{colour =}\NormalTok{ relDiff.frac)) }\SpecialCharTok{+}
  \FunctionTok{geom\_point}\NormalTok{() }\SpecialCharTok{+}
  \FunctionTok{theme}\NormalTok{(}\AttributeTok{legend.position =} \StringTok{\textquotesingle{}top\textquotesingle{}}\NormalTok{, }\AttributeTok{axis.text.x =} \FunctionTok{element\_text}\NormalTok{(}\AttributeTok{angle =} \DecValTok{90}\NormalTok{, }\AttributeTok{vjust =} \FloatTok{0.5}\NormalTok{, }\AttributeTok{hjust =} \DecValTok{1}\NormalTok{))}
\end{Highlighting}
\end{Shaded}

\begin{verbatim}
## Warning: Removed 14 rows containing missing values (`geom_point()`).
\end{verbatim}

\includegraphics{C:/Users/au583430/OneDrive - Aarhus universitet/Documents/GitHub/Pedersen-2023-app-digestate-NH3-review/lit_raw_vs_dig/stats/stats1_files/figure-latex/unnamed-chunk-3-1.pdf}

\begin{Shaded}
\begin{Highlighting}[]
\FunctionTok{ggplot}\NormalTok{(dw, }\FunctionTok{aes}\NormalTok{(pH.ref, pH.dig, }\AttributeTok{colour =}\NormalTok{ slurry.major.ref)) }\SpecialCharTok{+}
  \FunctionTok{geom\_point}\NormalTok{()}
\end{Highlighting}
\end{Shaded}

\begin{verbatim}
## Warning: Removed 8 rows containing missing values (`geom_point()`).
\end{verbatim}

\includegraphics{C:/Users/au583430/OneDrive - Aarhus universitet/Documents/GitHub/Pedersen-2023-app-digestate-NH3-review/lit_raw_vs_dig/stats/stats1_files/figure-latex/unnamed-chunk-3-2.pdf}

\begin{Shaded}
\begin{Highlighting}[]
\FunctionTok{ggplot}\NormalTok{(dw, }\FunctionTok{aes}\NormalTok{(pH.ref, dpH, }\AttributeTok{colour =}\NormalTok{ slurry.major.ref)) }\SpecialCharTok{+}
  \FunctionTok{geom\_point}\NormalTok{()}
\end{Highlighting}
\end{Shaded}

\begin{verbatim}
## Warning: Removed 8 rows containing missing values (`geom_point()`).
\end{verbatim}

\includegraphics{C:/Users/au583430/OneDrive - Aarhus universitet/Documents/GitHub/Pedersen-2023-app-digestate-NH3-review/lit_raw_vs_dig/stats/stats1_files/figure-latex/unnamed-chunk-3-3.pdf}

\hypertarget{stats}{%
\section{Stats}\label{stats}}

\begin{Shaded}
\begin{Highlighting}[]
\NormalTok{m1 }\OtherTok{\textless{}{-}} \FunctionTok{lm}\NormalTok{(pH.dig }\SpecialCharTok{\textasciitilde{}}\NormalTok{ pH.ref }\SpecialCharTok{+}\NormalTok{ slurry.major.ref, }\AttributeTok{data =}\NormalTok{ dw)}
\FunctionTok{summary}\NormalTok{(m1)}
\end{Highlighting}
\end{Shaded}

\begin{verbatim}
## 
## Call:
## lm(formula = pH.dig ~ pH.ref + slurry.major.ref, data = dw)
## 
## Residuals:
##      Min       1Q   Median       3Q      Max 
## -1.22736 -0.20649  0.03326  0.23480  0.91961 
## 
## Coefficients:
##                     Estimate Std. Error t value Pr(>|t|)   
## (Intercept)           5.9652     1.9543   3.052  0.00505 **
## pH.ref                0.2652     0.2659   0.997  0.32753   
## slurry.major.refPig   0.1378     0.1939   0.711  0.48317   
## ---
## Signif. codes:  0 '***' 0.001 '**' 0.01 '*' 0.05 '.' 0.1 ' ' 1
## 
## Residual standard error: 0.4623 on 27 degrees of freedom
##   (8 observations deleted due to missingness)
## Multiple R-squared:  0.09634,    Adjusted R-squared:  0.0294 
## F-statistic: 1.439 on 2 and 27 DF,  p-value: 0.2547
\end{verbatim}

\begin{Shaded}
\begin{Highlighting}[]
\NormalTok{m2 }\OtherTok{\textless{}{-}} \FunctionTok{lm}\NormalTok{(pH.dig }\SpecialCharTok{\textasciitilde{}}\NormalTok{ pH.ref, }\AttributeTok{data =}\NormalTok{ dw)}
\FunctionTok{summary}\NormalTok{(m2)}
\end{Highlighting}
\end{Shaded}

\begin{verbatim}
## 
## Call:
## lm(formula = pH.dig ~ pH.ref, data = dw)
## 
## Residuals:
##      Min       1Q   Median       3Q      Max 
## -1.28218 -0.17555  0.07506  0.27243  0.84633 
## 
## Coefficients:
##             Estimate Std. Error t value Pr(>|t|)   
## (Intercept)   5.3369     1.7277   3.089   0.0045 **
## pH.ref        0.3575     0.2300   1.554   0.1314   
## ---
## Signif. codes:  0 '***' 0.001 '**' 0.01 '*' 0.05 '.' 0.1 ' ' 1
## 
## Residual standard error: 0.4582 on 28 degrees of freedom
##   (8 observations deleted due to missingness)
## Multiple R-squared:  0.07942,    Adjusted R-squared:  0.04654 
## F-statistic: 2.416 on 1 and 28 DF,  p-value: 0.1314
\end{verbatim}

Try robust regression.

\begin{Shaded}
\begin{Highlighting}[]
\NormalTok{m3 }\OtherTok{\textless{}{-}}\NormalTok{ MASS}\SpecialCharTok{::}\FunctionTok{rlm}\NormalTok{(pH.dig }\SpecialCharTok{\textasciitilde{}}\NormalTok{ pH.ref }\SpecialCharTok{+}\NormalTok{ slurry.major.ref, }\AttributeTok{data =}\NormalTok{ dw)}
\FunctionTok{summary}\NormalTok{(m3)}
\end{Highlighting}
\end{Shaded}

\begin{verbatim}
## 
## Call: rlm(formula = pH.dig ~ pH.ref + slurry.major.ref, data = dw)
## Residuals:
##       Min        1Q    Median        3Q       Max 
## -1.226190 -0.255524  0.008057  0.208057  0.902945 
## 
## Coefficients:
##                     Value  Std. Error t value
## (Intercept)         5.3042 1.3889     3.8189 
## pH.ref              0.3543 0.1890     1.8750 
## slurry.major.refPig 0.1658 0.1378     1.2030 
## 
## Residual standard error: 0.3501 on 27 degrees of freedom
##   (8 observations deleted due to missingness)
\end{verbatim}

\begin{Shaded}
\begin{Highlighting}[]
\NormalTok{m4 }\OtherTok{\textless{}{-}}\NormalTok{ MASS}\SpecialCharTok{::}\FunctionTok{rlm}\NormalTok{(pH.dig }\SpecialCharTok{\textasciitilde{}}\NormalTok{ pH.ref, }\AttributeTok{data =}\NormalTok{ dw)}
\FunctionTok{summary}\NormalTok{(m4)}
\end{Highlighting}
\end{Shaded}

\begin{verbatim}
## 
## Call: rlm(formula = pH.dig ~ pH.ref, data = dw)
## Residuals:
##      Min       1Q   Median       3Q      Max 
## -1.30077 -0.19951  0.07298  0.22510  0.80426 
## 
## Coefficients:
##             Value  Std. Error t value
## (Intercept) 4.4867 1.2497     3.5902 
## pH.ref      0.4749 0.1664     2.8544 
## 
## Residual standard error: 0.3313 on 28 degrees of freedom
##   (8 observations deleted due to missingness)
\end{verbatim}

Digestate pH does seem correlated with raw pH but only with robust
regression. Issue seems to be a decrease in change in pH at higher raw
pH. So post digestion pH seems to be the same regardless of raw pH.
Seems plausible. Say low raw pH is caused by a lot of VFAs, which then
have no effect on digestate pH.

Simpler question: how does digestion change pH and DM?

\begin{Shaded}
\begin{Highlighting}[]
\FunctionTok{t.test}\NormalTok{(dw}\SpecialCharTok{$}\NormalTok{dpH)}
\end{Highlighting}
\end{Shaded}

\begin{verbatim}
## 
##  One Sample t-test
## 
## data:  dw$dpH
## t = 5.5511, df = 29, p-value = 5.492e-06
## alternative hypothesis: true mean is not equal to 0
## 95 percent confidence interval:
##  0.3258861 0.7061139
## sample estimates:
## mean of x 
##     0.516
\end{verbatim}

Clearly pH does increase, according to a one-sample t-test.
Mixed-effects model more appropriate.

\begin{Shaded}
\begin{Highlighting}[]
\NormalTok{m5pH }\OtherTok{\textless{}{-}} \FunctionTok{lmer}\NormalTok{(dpH }\SpecialCharTok{\textasciitilde{}}\NormalTok{ (}\DecValTok{1}\SpecialCharTok{|}\NormalTok{source), }\AttributeTok{data =}\NormalTok{ dw)}
\FunctionTok{summary}\NormalTok{(m5pH)}
\end{Highlighting}
\end{Shaded}

\begin{verbatim}
## Linear mixed model fit by REML ['lmerMod']
## Formula: dpH ~ (1 | source)
##    Data: dw
## 
## REML criterion at convergence: 32.6
## 
## Scaled residuals: 
##      Min       1Q   Median       3Q      Max 
## -2.40843 -0.24053  0.08254  0.28675  1.94922 
## 
## Random effects:
##  Groups   Name        Variance Std.Dev.
##  source   (Intercept) 0.30699  0.5541  
##  Residual             0.05266  0.2295  
## Number of obs: 30, groups:  source, 14
## 
## Fixed effects:
##             Estimate Std. Error t value
## (Intercept)   0.4670     0.1554   3.005
\end{verbatim}

\begin{Shaded}
\begin{Highlighting}[]
\FunctionTok{confint}\NormalTok{(m5pH)}
\end{Highlighting}
\end{Shaded}

\begin{verbatim}
## Computing profile confidence intervals ...
\end{verbatim}

\begin{verbatim}
##                 2.5 %    97.5 %
## .sig01      0.3426621 0.8435668
## .sigma      0.1673777 0.3470989
## (Intercept) 0.1502382 0.7812450
\end{verbatim}

Compare among animal types.

\begin{Shaded}
\begin{Highlighting}[]
\NormalTok{m6pH }\OtherTok{\textless{}{-}} \FunctionTok{lmer}\NormalTok{(dpH }\SpecialCharTok{\textasciitilde{}}\NormalTok{ slurry.major.ref }\SpecialCharTok{+}\NormalTok{  (}\DecValTok{1}\SpecialCharTok{|}\NormalTok{source), }\AttributeTok{data =}\NormalTok{ dw)}
\FunctionTok{summary}\NormalTok{(m6pH)}
\end{Highlighting}
\end{Shaded}

\begin{verbatim}
## Linear mixed model fit by REML ['lmerMod']
## Formula: dpH ~ slurry.major.ref + (1 | source)
##    Data: dw
## 
## REML criterion at convergence: 32.5
## 
## Scaled residuals: 
##     Min      1Q  Median      3Q     Max 
## -2.4381 -0.2670  0.1201  0.2397  1.9352 
## 
## Random effects:
##  Groups   Name        Variance Std.Dev.
##  source   (Intercept) 0.32301  0.5683  
##  Residual             0.05229  0.2287  
## Number of obs: 30, groups:  source, 14
## 
## Fixed effects:
##                     Estimate Std. Error t value
## (Intercept)           0.5678     0.2108   2.693
## slurry.major.refPig  -0.2347     0.3211  -0.731
## 
## Correlation of Fixed Effects:
##             (Intr)
## slrry.mjr.P -0.657
\end{verbatim}

Not clearly smaller for pig.

But we don't care about standard deviation estimates, so simpler to
explain mean by study.

\begin{Shaded}
\begin{Highlighting}[]
\NormalTok{dws }\OtherTok{\textless{}{-}} \FunctionTok{aggregate2}\NormalTok{(dw, }\FunctionTok{c}\NormalTok{(}\StringTok{\textquotesingle{}dpH\textquotesingle{}}\NormalTok{, }\StringTok{\textquotesingle{}dDM\textquotesingle{}}\NormalTok{), }\AttributeTok{by =} \StringTok{\textquotesingle{}source\textquotesingle{}}\NormalTok{, }\AttributeTok{FUN =} \FunctionTok{list}\NormalTok{(}\AttributeTok{mean =}\NormalTok{ mean, }\AttributeTok{n =} \ControlFlowTok{function}\NormalTok{(x) }\FunctionTok{sum}\NormalTok{(}\SpecialCharTok{!}\FunctionTok{is.na}\NormalTok{(x))))}
\end{Highlighting}
\end{Shaded}

\begin{Shaded}
\begin{Highlighting}[]
\FunctionTok{t.test}\NormalTok{(dws}\SpecialCharTok{$}\NormalTok{dpH.mean)}
\end{Highlighting}
\end{Shaded}

\begin{verbatim}
## 
##  One Sample t-test
## 
## data:  dws$dpH.mean
## t = 4.558, df = 11, p-value = 0.0008194
## alternative hypothesis: true mean is not equal to 0
## 95 percent confidence interval:
##  0.2907323 0.8337121
## sample estimates:
## mean of x 
## 0.5622222
\end{verbatim}

And repeat for DM:

\begin{Shaded}
\begin{Highlighting}[]
\FunctionTok{t.test}\NormalTok{(dw}\SpecialCharTok{$}\NormalTok{dDM)}
\end{Highlighting}
\end{Shaded}

\begin{verbatim}
## 
##  One Sample t-test
## 
## data:  dw$dDM
## t = -6.6101, df = 35, p-value = 1.224e-07
## alternative hypothesis: true mean is not equal to 0
## 95 percent confidence interval:
##  -2.909077 -1.542034
## sample estimates:
## mean of x 
## -2.225556
\end{verbatim}

Clearly DM does increase, according to a one-sample t-test.
Mixed-effects model more appropriate.

\begin{Shaded}
\begin{Highlighting}[]
\NormalTok{m5DM }\OtherTok{\textless{}{-}} \FunctionTok{lmer}\NormalTok{(dDM }\SpecialCharTok{\textasciitilde{}}\NormalTok{ (}\DecValTok{1}\SpecialCharTok{|}\NormalTok{source), }\AttributeTok{data =}\NormalTok{ dw)}
\FunctionTok{summary}\NormalTok{(m5DM)}
\end{Highlighting}
\end{Shaded}

\begin{verbatim}
## Linear mixed model fit by REML ['lmerMod']
## Formula: dDM ~ (1 | source)
##    Data: dw
## 
## REML criterion at convergence: 130.5
## 
## Scaled residuals: 
##     Min      1Q  Median      3Q     Max 
## -3.9984 -0.0775 -0.0079  0.1895  1.5780 
## 
## Random effects:
##  Groups   Name        Variance Std.Dev.
##  source   (Intercept) 2.7817   1.6679  
##  Residual             0.9728   0.9863  
## Number of obs: 36, groups:  source, 16
## 
## Fixed effects:
##             Estimate Std. Error t value
## (Intercept)  -2.1976     0.4555  -4.824
\end{verbatim}

\begin{Shaded}
\begin{Highlighting}[]
\FunctionTok{confint}\NormalTok{(m5DM)}
\end{Highlighting}
\end{Shaded}

\begin{verbatim}
## Computing profile confidence intervals ...
\end{verbatim}

\begin{verbatim}
##                  2.5 %    97.5 %
## .sig01       1.0683971  2.496403
## .sigma       0.7464718  1.379687
## (Intercept) -3.1143616 -1.277186
\end{verbatim}

Compare among animal types.

\begin{Shaded}
\begin{Highlighting}[]
\NormalTok{m6DM }\OtherTok{\textless{}{-}} \FunctionTok{lmer}\NormalTok{(dDM }\SpecialCharTok{\textasciitilde{}}\NormalTok{ slurry.major.ref }\SpecialCharTok{+}\NormalTok{  (}\DecValTok{1}\SpecialCharTok{|}\NormalTok{source), }\AttributeTok{data =}\NormalTok{ dw)}
\FunctionTok{summary}\NormalTok{(m6DM)}
\end{Highlighting}
\end{Shaded}

\begin{verbatim}
## Linear mixed model fit by REML ['lmerMod']
## Formula: dDM ~ slurry.major.ref + (1 | source)
##    Data: dw
## 
## REML criterion at convergence: 125.8
## 
## Scaled residuals: 
##     Min      1Q  Median      3Q     Max 
## -4.0063 -0.1773  0.0730  0.1991  1.6032 
## 
## Random effects:
##  Groups   Name        Variance Std.Dev.
##  source   (Intercept) 2.3977   1.5485  
##  Residual             0.9613   0.9805  
## Number of obs: 36, groups:  source, 16
## 
## Fixed effects:
##                     Estimate Std. Error t value
## (Intercept)          -2.7972     0.5416  -5.164
## slurry.major.refPig   1.5894     0.8828   1.800
## 
## Correlation of Fixed Effects:
##             (Intr)
## slrry.mjr.P -0.614
\end{verbatim}

Some evidence change is smaller for pigs.

We don't care about standard deviation estimates, so simpler to explain
mean by study.

\begin{Shaded}
\begin{Highlighting}[]
\FunctionTok{t.test}\NormalTok{(dws}\SpecialCharTok{$}\NormalTok{dDM.mean)}
\end{Highlighting}
\end{Shaded}

\begin{verbatim}
## 
##  One Sample t-test
## 
## data:  dws$dDM.mean
## t = -5.4297, df = 13, p-value = 0.0001152
## alternative hypothesis: true mean is not equal to 0
## 95 percent confidence interval:
##  -3.400268 -1.464613
## sample estimates:
## mean of x 
##  -2.43244
\end{verbatim}

\end{document}
